% !TeX root = ../defense.tex

\section{Evaluation}
\frame{\sectionpage}

\begin{frame}[fragile]{Linear Road Data}
Query Type, Time stamp, vehicle ID, speed, expressway, lane, direction, segment, position, query ID, start segment, end segment, day of week, minute of day, day in past 10 weeks
\begin{lstlisting}[caption= Example of Linear Road Data]
0,0,13,10,8,0,0,89,469920,-1,-1,-1,-1,-1,-1
0,0,17,10,8,0,1,65,348479,-1,-1,-1,-1,-1,-1
0,0,22,10,8,0,0,12,63360,-1,-1,-1,-1,-1,-1
0,0,33,10,8,0,1,94,501599,-1,-1,-1,-1,-1,-1
0,0,42,10,8,0,0,14,73920,-1,-1,-1,-1,-1,-1
0,0,4,10,7,0,0,61,322080,-1,-1,-1,-1,-1,-1
0,0,85,10,8,0,1,30,163679,-1,-1,-1,-1,-1,-1
0,0,11,10,6,0,1,41,221759,-1,-1,-1,-1,-1,-1
0,0,23,10,7,0,1,81,432959,-1,-1,-1,-1,-1,-1
0,0,15,10,6,0,0,5,26400,-1,-1,-1,-1,-1,-1
\end{lstlisting}
\end{frame}


\begin{frame}{Data Distribution}
    \begin{figure}
        \centering
        \includegraphics[scale=0.45]{totalb1.png}\\
        \caption{Shows for each ordering of operations, the number of data windows for which is was optimal ordering.}
        \label{fig:totalb1}
    \end{figure}
\end{frame}


\begin{frame}{Data Distribution}
    \begin{figure}
        \centering
        \includegraphics[scale=0.45]{trainingb1.png}\\
        \caption{Training data. Did a 70\% 30\% divide on data for training and testing respectively.}
        \label{fig:trainingb1}
    \end{figure}
\end{frame}

\begin{frame}{Data Distribution}
    \begin{figure}
        \centering
        \includegraphics[scale=0.45]{testingb1.png}\\
        \caption{Testing data. Did a 70\% 30\% divide on data for training and testing respectively.}
        \label{fig:testingb1}
    \end{figure}
\end{frame}

\begin{frame}[fragile]{Confusion Matrix}
    \begin{lstlisting}[caption=Confusion matrix for DQN classificaiton, label={lst:confusion_matrix}]
        [[   0    0    0    0    0    0    0    0    0    0    0    1    0]
         [   0    0    0    0    0    0    0    0    0    0    0   16   13]
         [   0    0    0    0    0    0    0    0    0    0    0   18    5]
         [   0    0    0    0    0    0    0    0    0    0    0   13   12]
         [   0    0    0    0    0    0    0    0    0    0    0   12   13]
         [   0    0    0    0    0    0    0    0    0    0    0   17    6]
         [   0    0    0    0    0    0    0    0    0    0    0    9    9]
         [   0    0    0    0    0    0    0    0    0    0   10   21   49]
         [   0    0    0    0    0    0    0    0    0    3    4   13   26]
         [   0    0    0    0    0    0    0    1    0   18   77  213  374]
         [   0    0    0    0    0    0    0    0    0   19   69  118  195]
         [   0    0    2    0    0    0    0    0    0  116  406 6347 6167]
         [   0    0    1    0    0    0    0    6    0  113  476 6338 6355]]
    \end{lstlisting}
\end{frame}

\begin{frame}{Confusion Matrix}
    The predictions on trained model lead to the following confusion matrix.
    \begin{figure}
        \centering
        \includegraphics[scale=0.3]{cm3.png}\\
        \caption{DQN predictions visualized as confusion matrix}
        \label{fig:dqn_r2_1}
    \end{figure}
\end{frame}

\begin{frame}{Prediction 1}
    For first data window, we have the following number of operations required.
    \begin{figure}
        \centering
        \includegraphics[scale=0.4]{operations1.png}\\
        \caption{The figure shows the log$_{10}$(number of operations) required to execute the query depending on the ordering of the selection operators chosen. The predicted optimal ordering is shown in red.}
        \label{fig:operations1}
    \end{figure}
\end{frame}

\begin{frame}{Prediction 2}
    For second data window, we have the following number of operations required.
    \begin{figure}
        \centering
        \includegraphics[scale=0.4]{operations2.png}\\
        \caption{The figure shows the log$_{10}$(number of operations) required to execute the query depending on the ordering of the selection operators chosen. The predicted optimal ordering is shown in red.}
        \label{fig:operations2}
    \end{figure}
\end{frame}

\begin{frame}{Overall accuracy}
    The model predicted optimal move correctly for $12978$ data windows, out of $27681$ data windows, i.e. $47\%$.\\
\end{frame}

\begin{frame}{Comparison}
    If the DSMS executed 
    \begin{enumerate}
        \item The TRUE optimal ordering, for each data window, on the given test data set, it will require 890M operations overall.
        \item The PREDICTED optimal ordering, for each data window, on the given test data, it will required 900M operations overall.
        \item The WORST ordering, for each data window, on the given test data, it will require 2310M operations overall.
    \end{enumerate}
\end{frame}

