\chapter{Introduction}
\label{chapter:Introduction}
\thispagestyle{myheadings}

\section{Motivation}
\label{sec:Motivation}
In recent years, the ability to gather data has increased tremendously due to various sensory devices and the cheap cost of data storeage. Some examples of data gathering and generating sources are,
\begin{itemize}
\item \textbf{Finance}, Stock market, the continuous flow includes the price of any given stock at any time. 
\item \textbf{Network Management}, Network administrators need to continuously monitor the traffic flow on their network as they need to help maintain a level of Quality of Service for their clients.
\item \textbf{Healthcare}, Monitoring the vitals of the patients is an important and continuous process. 
\item \textbf{National Security} Monitoring airspace, getting continuous information from various sources and inferring results form them.
\end{itemize} 
For these examples it is clear that there exists a need to continously keep processing data stream and reporting back various statistics and any abnormality. But this continuous monitoring and reporting is difficult to do on tradiational Data Base Management Systems(DBMS). A different approach has to be adapted to meet the demands in various industries.
\par Stream optimization in a certain sense is modification of the stream data to make the quering process faster. Few motivations for this include, to be able to make use of opportunities presented by faster calculation, mitiage risks before it is late, keeping the latest information readily available in understandable format.

\section{Problem at hand}
\label{sec:Problem at hand}
\par The increase in data gathering capabilities and speed need to be accompanied with increase in speed at which data can be analysed. There are various challenged that come up when addressing this, such as High frequency of the updates and reporting, buffer overflows, overheads, context-switches during processing and many more. The motivation for our approach is, most of the query optimization methods do not look at the data itself and rather try to come up with a very general optimization which can be applied to any query. Many times an indepth look at the data might prove to be rather expensive. In this paper we try to detect trends in the data and help optimize quering using those and test our model on benchmark cases.

\section{Structure of thesis}
\label{sec:Structure of thesis}
The next chapter gives an in-depth view of the pipeline used by the current state of art technology for query optimization in traditional data bases including the mathematical knowledge for simplification and the overall framework. The next chapter also introduces the reader to data stream and how data bases are used for them called DSMS and showcases an approach to optimize queries on data streams for the problem discusses above. The following chapter list out the details of implementation, challenges face, evaulation methods used, benchmark test case timings, followed by a summary of the paper.

\section{Conclusion}
\label{sec:Conclusion}
With the help of this thesis, we hope to understand how query processing works in Data base management systems, the challenges they face, the way Data Stream Management Systems work, their challenges. We also propose a method for query optimization which will look at windows of data(defined later in the paper). The optimizations for the quering process will be tested against various test cases and compared with existing algorithms.
