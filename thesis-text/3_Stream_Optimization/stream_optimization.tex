\chapter{Stream Optimization}
\label{chapter:stream_optimization}
\thispagestyle{myheadings}

% set this to the location of the figures for this chapter. it may
% also want to be ../Figures/2_Body/ or something. make sure that
% it has a trailing directory separator (i.e., '/')!
\graphicspath{{3_Conclusion/Figures/}}

\section{Formal Problem statement}
To define a concrete problem statement, we will need to decide upon
\begin{itemize}
    \item The learning data, which will work as the input to our model, the data should be replicable so that model can be tested consistently.
    \item While more implementation based, the method to take input needs to be decided as well and maintained for consistancy reasons.
    \item After fixing the input data and method, for DSMS need to decide upon queries of sufficient complexity for the results to be interesting. 
    Only by having sufficiently complex queries will the query graph lead to interesting results.
    \item At this point need to combine the knowledge from chapter 2 regarding querying processing to come up with a prediction model, its execution, training and evaluation.

\end{itemize}

% 1. What is the stream query optimization? 
% We can have complex queries. Conctinous queries because it runs on top of the data stream. 
% They include join, selection, projection, user-defined functions, etc. 
% 
% 2. If we machine learning what is for us the training data. 
% We need to know about history of different runs of queries. 
% Query-1 can have different execution plans. {(P1, Cost1), (P2, Cost2), ... (Pn, Const_n)}. There is a min cost. 
% Our query does not match one-to-one to the past queries. How we can use parts of the past? 
% Query-1 comes with features of {f1, f2, ..., fn}, we have an optimized plan for this query. Things like graphs structures. 
% Query-future , you can extract features from it {f1, f2, ..., fn}

% Problem is almost a similarity matching problem. 
