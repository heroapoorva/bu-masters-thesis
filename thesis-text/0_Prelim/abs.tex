% ABSTRACT
Our increasingly connected world has resulted in the generation of many data streams. With the rise in network traffic and increased interconnectivity, analyzing data and extracting key information has become challenging. Executing queries to extract insights from the data is now more essential than ever to make quick and efficient decisions. The flexibility of reinforcement learning, along with strong prediction properties, and precise mathematical model of Deep neural networks combined results in the technique called Deep reinforcement learning which can be a robust and promising method to reduce query processing times. While there is a lot of research on query optimization on data streams, none showcase the use of Deep reinforcement learning. The continuous nature of datastreams renders the conventional approaches inapplicable, due to increased processing time. The method implemented in this thesis is generic enough so that it applies to a wide range of data streams and queries to optimize the query processing time. The goal of this study is to act as a proof of concept. The thesis explores the use of deep reinforcement learning on a query from Linear road benchmark data. The proposed method is demonstrated to be effective in picking optimal query plan, and reducing the query execution time. We hope the work in this thesis encourages others to work on the development of a robust and more generalized system, capable of optimizing any query on any data stream.

