\chapter{Implementation}
\label{chapter:implementation}
\thispagestyle{myheadings}

% set this to the location of the figures for this chapter. it may
% also want to be ../Figures/2_Body/ or something. make sure that
% it has a trailing directory separator (i.e., '/')!
\graphicspath{}
This chapter explores the steup required to conduct the experiments. The chapter is divided into data generation, query execution and lastly deep reinforcement learning for move prediction.

\section{Data Generation}
The data is generated using the walmart linear road found \cite{walmart_linearoad}

\begin{lstlisting}[language=bash]
java com.walmart.linearroad.generator.LinearGen [-o <output file>] [-x <number of xways>] [-m <dummy value to activate multi-threading>]
\end{lstlisting}

\subsection{Schema}
The above generated data can be interpretated as follows:-
\begin{center}
\begin{tabular}{ |c|c| } 
 \hline
  Column$1$ & Tells the type of query. \\  
 \hline
  Column$2$ & Timestamp position. \\  
 \hline
  Column$3$ & Vehicle identification number\\  
 \hline
  Column$4$ & Speed of the vehicle \\  
 \hline
  Column$5$ & Express way number \\  
 \hline
  Column$6$ & Lane ID $(0,...,4)$\\  
 \hline
  Column$7$ & Direction of movement($0=$East or $1=$West) \\  
 \hline
  Column$8$ & Segment ID $(0,...,99)$ \\
 \hline
  Column$9$ & Position of the vehicle. \\  
 \hline
  Column$10$ & Query identifier \\  
 \hline
  Column$11$ & Start Segment \\  
 \hline
  Column$12$ & End Segment \\  
 \hline
  Column$13$ & Day of the week \\  
 \hline
  Column$14$ & Minute of the day \\  
 \hline
  Column$15$ & Day in the past $10$ weeks \\  
 \hline
\end{tabular}
\end{center}
